\documentclass[compress,mathserif,cjk]{beamer}
\usepackage{mathrsfs}
\usepackage{color}
\usepackage{CJK}
\usepackage{amssymb}
\usepackage{amsmath}
\usepackage{extarrows}
\usepackage{ulem}
\usepackage{latexsym}

% Copyright 2003 by Till Tantau <tantau@cs.tu-berlin.de>.
%
% This program can be redistributed and/or modified under the terms
% of the LaTeX Project Public License Distributed from CTAN
% archives in directory macros/latex/base/lppl.txt.

%
% The purpose of this example is to show how \part can be used to
% organize a lecture.
%
\usetheme{Warsaw}
  % 可供选择的主题参见 beameruserguide.pdf, 第 134 页起
  % 无导航条的主题: Bergen, Boadilla, Madrid, Pittsburgh, Rochester;
  % 有树形导航条的主题: Antibes, JuanLesPins, Montpellier;
  % 有目录竖条的主题: Berkeley, PaloAlto, Goettingen, Marburg, Hannover;
  % 有圆点导航条的主题: Berlin, Dresden, Darmstadt, Frankfurt, Singapore, Szeged;
  % 有节与小节导航条的主题: Copenhagen, Luebeck, Malmos, Warsaw

%  \setbeamercovered{transparent}
% 如果取消上一行的注解 %, 就会使得被覆盖部分变得透明(依稀可见)

\usepackage[english]{babel}
\usepackage[latin1]{inputenc}
\usepackage{CJK}

%\usepackage{beamerthemesplit}
%\usepackage{beamerthemeshadow,color}
\usepackage{pgf,pgfarrows,pgfnodes,pgfautomata,pgfheaps,pgfshade}
\usepackage{amsmath,amssymb}
\usepackage{bm}
\usepackage{colortbl}

\graphicspath{{images/}}         %% 图片路径. 本文的图片都放在这个文件夹里了.
\DeclareGraphicsRule{*}{mps}{*}{} %% 使 pdflatex 可以纳入 metapost 做的图片.
\renewcommand{\div}{\operatorname{div}}
\renewcommand{\raggedright}{\leftskip=0pt \rightskip=0pt plus 0cm}
\raggedright %% 中文对齐

%==============自定义: 逐个 item 高亮(\hilite), 或"高黑"(\hidark)==================%
\def\hilite<#1>{%
\temporal<#1>{\color{blue!35}}{\color{magenta}}%
{\color{blue!75}}}
\def\hidark<#1>{%
\temporal<#1>{\color{black!35}}{\color{magenta}}%
{\color{black}}}

\newcolumntype{H}{>{\columncolor{blue!20}}c!{\vrule}}
\newcolumntype{H}{>{\columncolor{blue!20}}c}  %% 表格设置
%==================================参考文献==============================================================
\newcommand{\upcite}[1]{\textsuperscript{\cite{#1}}}  %自定义命令\upcite, 使参考文献引用以上标出现
\bibliographystyle{plain}
%=========================================================================================

\def\colorb{\textcolor[rgb]{0.00,0.00,1.00}}
\def\colorg{\textcolor[rgb]{0.00,1.00,0.00}}
\def\colorr{\textcolor[rgb]{1.00,0.00,0.00}}
\newcommand\hdim{\dim_{\mathrm H}}
\newtheorem{lem}{Lemma}[section]
\newtheorem{nt}[lem]{Notation}
\newtheorem{dfn}[lem]{Definition}
\newtheorem{pro}[lem]{Proposition}
\newtheorem{thm}[lem]{Theorem}
\newtheorem{exa}[lem]{Example}
\newtheorem{cor}[lem]{Corollary}
\theoremstyle{remark}
\newtheorem*{rem}{Remark}
\numberwithin{equation}{section}
\def\N{\mathbb N}
\def\Q{\mathbb Q}
\def\R{\mathbb R}
\def\Z{\mathbb Z}
\def\vep{\varepsilon}
%=========================================================================================
%\setbeamercovered{transparent}
%
% The following info should normally be given in you main file:
%

%%%%%%%%%%%%%%%%%%%%%%%%%%%%%%%%%%%%%%%%%%%%%%%%%%%%%%%%%%%%%%%%%%%%%%%%%%%%%%%%%%%%%%%%%%%%%%%%%%%%%%%%%
%                                           定制幻灯片---重定义字体、字号命令                           %
%%%%%%%%%%%%%%%%%%%%%%%%%%%%%%%%%%%%%%%%%%%%%%%%%%%%%%%%%%%%%%%%%%%%%%%%%%%%%%%%%%%%%%%%%%%%%%%%%%%%%%%%%
\newcommand{\song}{\CJKfamily{song}}    % 宋体   (Windows自带simsun.ttf)
\newcommand{\fs}{\CJKfamily{fs}}        % 仿宋体 (Windows自带simfs.ttf)
\newcommand{\kai}{\CJKfamily{kai}}      % 楷体   (Windows自带simkai.ttf)
\newcommand{\hei}{\bf}      % 黑体   (Windows自带simhei.ttf)
\newcommand{\li}{\CJKfamily{li}}        % 隶书   (Windows自带simli.ttf)
\newcommand{\you}{\CJKfamily{you}}      % 幼圆   (Windows自带simyou.ttf)
\newcommand{\chuhao}{\fontsize{42pt}{\baselineskip}\selectfont}     % 字号设置
\newcommand{\xiaochuhao}{\fontsize{36pt}{\baselineskip}\selectfont} % 字号设置
\newcommand{\yichu}{\fontsize{32pt}{\baselineskip}\selectfont}      % 字号设置
\newcommand{\yihao}{\fontsize{28pt}{\baselineskip}\selectfont}      % 字号设置
\newcommand{\erhao}{\fontsize{21pt}{\baselineskip}\selectfont}      % 字号设置
\newcommand{\xiaoerhao}{\fontsize{18pt}{\baselineskip}\selectfont}  % 字号设置
\newcommand{\sanhao}{\fontsize{15.75pt}{\baselineskip}\selectfont}  % 字号设置
\newcommand{\sihao}{\fontsize{14pt}{\baselineskip}\selectfont}      % 字号设置
\newcommand{\xiaosihao}{\fontsize{12pt}{\baselineskip}\selectfont}  % 字号设置
\newcommand{\wuhao}{\fontsize{10.5pt}{\baselineskip}\selectfont}    % 字号设置
\newcommand{\xiaowuhao}{\fontsize{9pt}{\baselineskip}\selectfont}   % 字号设置
\newcommand{\liuhao}{\fontsize{7.875pt}{\baselineskip}\selectfont}  % 字号设置
\newcommand{\qihao}{\fontsize{5.25pt}{\baselineskip}\selectfont}    % 字号设置

%======================= 标题名称中文化 ============================%
\newtheorem{dingyi}{\hei 定义~}[section]
\newtheorem{dingli}{\hei 定理~}[section]
\newtheorem{yinli}[dingli]{\hei 引理~}
\newtheorem{tuilun}[dingli]{\hei 推论~}
\newtheorem{mingti}[dingli]{\hei 命题~}
%%%%%%%%%%%%%%%%%%%%%%%%%%%%%%%%%%%%%%%%%%%%%%%%%%%%%%%%%%%%%%%%%%%%%%%%%%%%%%%%%%%%%%%

% \usepackage{beamerthemesplit} // Activate for custom appearance


\title{\textsc{第1章\ \ \ 向量代数、空间中直线与平面}}
\author{郑州大学数学与统计学院 线性代数教研室}
\date{}

\begin{document}
\begin{CJK}{UTF8}{gbsn}
\frame{\titlepage}

\begin{frame}\frametitle{目录}
 \tableofcontents
\end{frame}
%%%%%%%%%%%%%%%%%%%%%%%%%%%%%%%%%%%%%%%%%%%%%%%%%%%%%%%%%%%%%%%%%%%%%%%%%%%%%%%%%%%%%%

\section[1.1]{1.1 空间直角坐标系}
\begin{frame}{空间直角坐标系}
 \setlength{\unitlength}{2cm}\begin{center}
 \begin{picture}(1,1)
 \put(0,0){\vector(0,1){1}}
 \put(0,0){\vector(1,0){1.2}}
 \put(0,0){\vector(-1,-1){0.5}}

 \put(0,-0.1){$o$}
 \scriptsize{
 \put(0.95,0.45){$\bullet$}
 \put(-0.6,-0.5){$x$}\put(1.15,-0.15){$y$}\put(-0.12,0.94){$z$}\put(0.9,0.55){$P(x,y,z)$}
 }
 \end{picture}\end{center}
 \pause\vskip 25pt
 \ \ \ \ 空间中任意一点$P$都对应一个有序三元组$(x,y,z)$,称为$P$的{\hei 坐标},记为$P(x,y,z)$.
 \pause\vskip 10pt
 \ \ \ \ 空间中两点$P_1(x_1,y_1,z_1),P_2(x_2,y_2,z_2)$的距离为
 $$|P_1P_2|=\sqrt{(x_2-x_1)^2+(y_2-y_1)^2+(z_2-z_1)^2}.$$
\end{frame}

\section[1.2]{1.2 向量的概念}
\begin{frame}{向量的概念}
 \ \ \ \ 只有大小的量称为{\hei 数量}(或{\hei 标量});
 \vskip 5pt
 \ \ \ \ 既有大小又有方向的量称为{\hei 向量}(或{\hei 矢量})。
 \pause\vskip 1pt
 \ \ \ \ 向量$\bm a$可以用一个带箭头的有向线段$\overset{\longrightarrow}{AB}$来表示 \hskip 5pt
 \setlength{\unitlength}{2cm}
\put(0.2,0.2){\vector(1,0){1.2}}
  {\scriptsize
  \put(0.15,0.05){$A$}\put(1.3,0.05){$B$}
  }
 \vskip 1pt
 线段$AB$的长度称为向量$a$的{\hei 长度}(或{\hei 大小}或{\hei 模}),记为$|\bm a|$或$|\overset{\longrightarrow}{AB}|$.
 \pause \vskip 10pt
 \ \ \ \ 特殊向量:{\hei 单位向量(幺矢)},{\hei 零向量}$\bm0$(方向不定),{\hei负向量}$-\bm a$.
 \vskip 10pt
 \pause\ \ \ \ 向量{\hei 相等}:长度相等且方向相同,与起点、终点无关。
 \vskip 10pt
 \ \ \ \ 这里所说的向量都是{\hei 自由向量},可以平行移动。若一组向量通过平移可以共线(或共面),则称这组向量{\hei 共线}({\hei 或共面})。
 \vskip 10pt
 \ \ \ \ 共线向量又称为{\hei 平行向量},记为$\bm a//\bm b$.

\end{frame}

\section[1.3]{1.3 向量的线性运算}
\begin{frame}{向量的加法、减法}
 $\bullet$ 向量的加法
 \vskip 5pt
 \ \ \ \ (1) 三角形法则\\
 \hspace{8em}$\bm a+\bm b=\overset{\longrightarrow}{AB}+\overset{\longrightarrow}{BC}=\overset{\longrightarrow}{AC}.$
  \vskip 10pt
 \pause
 \ \ \ \ (2) 平行四边形法则\\
 \hspace{8em}$\bm a+\bm b=\overset{\longrightarrow}{AB}+\overset{\longrightarrow}{AC}=\overset{\longrightarrow}{AD}.$
 \vskip 10pt\pause $\bullet$ 向量的减法
  $$\bm a-\bm b=\bm a+(-\bm b).$$
 \setlength{\unitlength}{2cm}\begin{center}
 \begin{picture}(4,1)
 \put(-0.5,0){\vector(1,0){2}}
  \put(-0.5,0){\vector(1,1){0.8}}\put(0.3,0.8){\vector(3,-2){1.2}}
  {\scriptsize
  \put(-0.6,-0.15){$A$}\put(1.4,-0.15){$C$}\put(0.25,0.85){$B$}
  \put(-0.3,0.35){$\bm a$}\put(1.,0.4){$\bm b$}
  }
  \put(2.2,0){\vector(1,0){1.8}}\put(2.2,0){\vector(3,4){0.6}}\put(2.8,0.8){\line(1,0){1.8}}\put(4,0){\line(3,4){0.6}}
  \put(2.2,0){\vector(3,1){2.4}}\put(4,0){\vector(-3,2){1.2}}
  {\scriptsize
  \put(2.1,-0.15){$A$}\put(4,-0.15){$C$}\put(2.7,0.85){$B$}\put(4.6,0.85){$D$}
  \put(2.35,0.35){$\bm a$}\put(3,-0.15){$\bm b$}\put(3.8,0.44){$\bm a+\bm b$}\put(3.08,0.63){$\bm a-\bm b$}
  }
 \end{picture}\end{center}
\end{frame}

\begin{frame}{向量的数乘}
 \ \ \ \ 实数$\lambda$与向量$\bm a$的乘积,称为{\hei 数乘向量},记为$\lambda\bm a$,它的长度为$|\lambda\bm a|=|\lambda||\bm a|$,其方向当$\lambda>0$时与$\bm a$同向;当$\lambda<0$时与$\bm a$ 反向。
 \vskip 5pt
 \ \ \ \ 这种运算称为向量的{\hei 数乘运算}或向量的{\hei 数量乘法}。
 \pause\vskip 5pt
 \ \ \ \ 一些结论:
 \vskip 5pt
 \ \ \ \ $\bullet~~~\lambda\bm a=0~\Longleftrightarrow~\lambda=0$或$\bm a=0$;
 \vskip 5pt
 \ \ \ \ $\bullet~~~\bm a//\bm b~\Longleftrightarrow~\bm a=\lambda\bm b$;
 \vskip 5pt
 \ \ \ \ $\bullet~~~1\bm a=\bm a,~~(-1)\bm a=-\bm a$;
  \vskip 5pt
 \ \ \ \ $\bullet~~~\lambda(\mu\bm a)=(\lambda\mu)\bm a$;
 \vskip 5pt
 \ \ \ \ $\bullet~~~\lambda(\bm a+\bm b)=\lambda\bm a+\lambda\bm b,~~(\lambda+\mu)\bm a=\lambda\bm a+\mu\bm a$.
 \pause\vskip 10pt
 \ \ \ \ {\hei 例~1} 设$\bm a\neq0$,则$\bm a_0=\frac{1}{|\bm a|}\bm a$是唯一与$\bm a$同方向的单位向量,且$\bm a=|\bm a|\bm a_0$.
\end{frame}

\section[1.4]{1.4 向量的数量积、向量积、混合积}
\begin{frame}{一、数量积}
 \ \ \ \ {\hei 定义~1.1} 向量$\bm a$与$\bm b$的{\hei 数量积}(或{\hei 内积}) $\bm a\cdot\bm b$ 定义为
 $$\bm a\cdot\bm b=|\bm a||\bm b|\cos\langle\bm a,\bm b\rangle,$$
 其中$\langle\bm a,\bm b\rangle$为$\bm a$与$\bm b$的夹角($0\leq\langle\bm a,\bm b\rangle\leq\pi)$.
 \vskip 10pt
 \ \ \ \ 几何意义:$\bm a\cdot\bm b$表示$|\bm a|$乘以另一向量$\bm b$在$\bm a$上的投影$|\bm b|\cos\langle\bm a,\bm b\rangle$.
 \pause \vskip 5pt
 \ \ \ \ 一些结论:
 \vskip 2pt
 \ \ \ (1) 单位向量$\bm a_0,\bm b_0$的内积$\bm a_0\cdot\bm b_0=\cos\langle\bm a_0,\bm b_0\rangle,~\bm a_0\cdot\bm b=|\bm b|\cos\langle\bm a_0,\bm b\rangle$;
  \vskip 2pt
 \ \ \ (2) $\bm a\cdot\bm b=0~\Longleftrightarrow~\bm a\bot\bm b,$ 规定零向量$\bm 0$与任意向量都垂直;
 \vskip 2pt
 \ \ \ (3) 内积的正负取决于向量夹角是锐角或钝角;
 \vskip 5pt
 \ \ \ (4) $\bm a\cdot\bm a=|\bm a|^2.$
 \vskip 5pt
 \pause\ \ \ \ 运算规律:
  \vskip 2pt
 \ \ \ \ $\bm a\cdot\bm b=\bm b\cdot\bm a,~~~(\lambda\bm a)\cdot\bm b=\lambda(\bm a\cdot\bm b),~~~\bm a\cdot(\bm b+\bm c)=\bm a\cdot\bm b+\bm a\cdot\bm c.$
\end{frame}

\begin{frame}{二、向量积}
 \ \ \ \ {\hei 定义~1.2} 向量$\bm a$与$\bm b$的{\hei 向量积}(或{\hei 矢量积}) $\bm a\times\bm b$ 定义为
 \vskip 5pt
 \ \ \ \ 其大小为~$|\bm a\times\bm b|=|\bm a||\bm b|\sin\langle\bm a,\bm b\rangle$;
 \vskip 5pt
 \ \ \ \ 其方向与~$\bm a,\bm b$~都垂直,且与~$\bm a,\bm b$~构成右手系。
 \pause\vskip 10pt
 \ \ \ \ 一些结论:
 \vskip 5pt
 \ \ \ \ (1) $\bm a\times\bm b$的长度等于以~$\bm a,\bm b$~为边的平等四边形的面积;
 \vskip 5pt
 \ \ \ \ (2) 若单位向量$\bm e$与$\bm a\times\bm b$同向,则~$\bm a\times\bm b=|\bm a||\bm b|\sin\langle\bm a,\bm b\rangle\bm e$;
 \vskip 5pt
 \ \ \ \ (3) $\bm a\times\bm b=0~\Longleftrightarrow~\bm a//\bm b,~$规定零向量$\bm 0$与任意向量都平行;
 \vskip 5pt
 \ \ \ \ (4) $\bm a\times\bm a=0$.
 \vskip 10pt
 \pause\ \ \ \ 运算规律:
 \vskip 5pt
 $\bm a\times\bm b=-\bm b\times\bm a,~~(\lambda\bm a)\times\bm b=\lambda(\bm a\times\bm b),~~\bm a\times(\bm b+\bm c)=\bm a\times\bm b+\bm a\times\bm c.$
\end{frame}

\begin{frame}{三、混合积}
 \ \ \ \ {\hei 定义~1.3} 设$\bm a,\bm b ,\bm c$为三个向量,则称$(\bm a\times\bm b)\cdot\bm c$为$\bm a,\bm b ,\bm c$的{\hei 混合积},记为$(\bm a,\bm b ,\bm c)$.
 \vskip 10pt
 \ \ \ \ 几何意义:若$\bm a,\bm b ,\bm c$不共面,则$|(\bm a,\bm b ,\bm c)|$恰是以$\bm a,\bm b ,\bm c$为棱的平行六面体的体积,当$\bm a,\bm b ,\bm c$作成右手系时$(\bm a,\bm b ,\bm c)$取正号;作成左手系里$(\bm a,\bm b ,\bm c)$取负号。
 \vskip 10pt
 \pause\ \ \ \ 一些结论:
 \vskip 5pt
 \ \ \ \ (1) $(\bm a,\bm b ,\bm c)=0~\Longleftrightarrow~\bm a,\bm b ,\bm c$共面(含共线的特殊情形);
 \vskip 5pt
 \ \ \ \ (2) $(\bm a,\bm b ,\bm c)=(\bm b ,\bm c,\bm a)=(\bm c,\bm a,\bm b)$;
 \vskip 5pt
 \ \ \ \ (3) $(\bm a\times\bm b)\cdot\bm c=\bm a\cdot(\bm b\times\bm c).$
\end{frame}

\begin{frame}
 \ \ \ \ {\hei 例~3} 设$\bm a,\bm b ,\bm c$为三个不共面的向量,$\bm d$是任意向量,求实数$\lambda,\mu,\gamma$使$\bm d=\lambda\bm a+\mu\bm b+\gamma\bm c$.
 \pause\vskip 10pt
 \ \ \ \ {\hei 解} 已知存在实数$\lambda,\mu,\gamma$使$\bm d=\lambda\bm a+\mu\bm b+\gamma\bm c.$
 两边用$\bm b\times\bm c$作内积,得
 \begin{eqnarray*}
 (\bm b\times\bm c)\cdot\bm d &=& (\bm b\times\bm c)\cdot(\lambda\bm a+\mu\bm b+\gamma\bm c) \\
 &=& \lambda(\bm b,\bm c,\bm a)\\
 &=& \lambda(\bm a,\bm b,\bm c).
 \end{eqnarray*}
 因为$\bm a,\bm b ,\bm c$不共面,所以$(\bm a,\bm b,\bm c)\neq0$. 从而得
 $$\lambda=\frac{(\bm b,\bm c,\bm d)}{(\bm a,\bm b,\bm c)}.$$
 同理可得
 $$\mu=\frac{(\bm c,\bm a,\bm d)}{(\bm a,\bm b,\bm c)},~~~\gamma=\frac{(\bm a,\bm b,\bm d)}{(\bm a,\bm b,\bm c)}.$$
\end{frame}
\section[1.5]{1.5 向量的坐标}

\begin{frame}\frametitle{一、向量的坐标}
 \setlength{\unitlength}{2cm}\begin{center}
 \begin{picture}(1,1)
 \put(0,0){\vector(0,1){1}}
 \put(0,0){\vector(1,0){1.5}}
 \put(0,0){\vector(-1,-1){0.5}}
 \put(0,0){\vector(2,1){0.95}}
 \put(0,0){\vector(0,1){0.3}}
 \put(0,0){\vector(1,0){0.3}}
 \put(0,0){\vector(-1,-1){0.15}}
 \put(0.95,-0.3){\vector(0,1){0.78}}
 \put(-0.3,-0.3){\vector(1,0){1.25}}
 \put(0,0){\vector(-1,-1){0.3}}
 \put(0,-0.1){$o$}
 \scriptsize{
 \put(-0.2,-0.1){$\bm{i}$}\put(0.2,-0.15){$\bm{j}$}\put(-0.6,-0.5){$x$}\put(-0.12,0.15){$\bm{k}$}\put(1.45,-0.15){$y$}\put(-0.12,0.94){$z$}
 \put(-0.45,-0.25){$x\bm{i}$}\put(0.3,-0.42){$y\bm{j}$}\put(0.98,0.1){$z\bm{k}$}\put(0.4,0.25){$\bm{r}$}\put(0.9,0.55){$P(x,y,z)$}
 }
 \end{picture}\end{center}
 \vskip 25pt
 空间中的点集与向量集一一对应,即
 $$P\in V_3\overset{1-1}\longleftrightarrow\overset{\longrightarrow}{OP}$$
 这里,我们称$\overset{\longrightarrow}{OP}$为$P$点的{\hei 径向}或{\hei 定位向量}。
 \pause 令$\bm{r}=\overset{\longrightarrow}{OP}$,则由三角形法则得
 $$\bm r=\pause x\bm i+y\bm j+z\bm k.$$
 此时称$P$点的坐标$(x,y,z)$为$\bm r$的{\hei 坐标},又称$x\bm i,y\bm j,z\bm k$为$\bm r$沿$x$轴,$y$轴,$z$轴的{\hei 分量}。因此不加区别地写成$\bm r=(x,y,z)$.

\end{frame}
%===========================================frame1========================================
\begin{frame}
 设$\overset{\longrightarrow}{OP_1}=(x_1,y_1,z_1),~\overset{\longrightarrow}{OP_2}=(x_2,y_2,z_2)$,则
 $$\overset{\longrightarrow}{P_1P_2}=\pause\overset{\longrightarrow}{OP_2}-\overset{\longrightarrow}{OP_1}=(x_2-x_1)\bm i+(y_2-y_1)\bm j+(z_2-z_1)\bm k,$$
 因此向量$\overset{\longrightarrow}{P_1P_2}$的坐标等于终点坐标减去始点的坐标,即
 $$\overset{\longrightarrow}{P_1P_2}=(x_2-x_1,y_2-y_1,z_2-z_1).$$

\end{frame}

%-------------------------------------------------frame3-------------------------------------------------------
\begin{frame}\frametitle{二、向量的坐标运算}

 (1) 向量加(减)法的坐标运算
 \vskip 5pt
 设$\bm a_1=(x_1,y_1,z_1),~\bm a_2=(x_2,y_2,z_2)$,则
 $$\bm a_1\pm\bm a_2=\pause(x_1\pm x_2)\bm i+(y_1\pm y_2)\bm j+(z_1\pm z_2)\bm k,$$
 即$\bm a_1\pm\bm a_2=(x_1\pm x_2,y_1\pm y_2,z_1\pm z_2).$
 \pause\vskip 10pt
 (2) 向量数乘(即数乘向量)的坐标运算
 \vskip 5pt
 设$\bm a=(x,y,z),~\lambda$为实数,则
 $$\lambda\bm a=\pause(\lambda x)\bm i+(\lambda y)\bm j+(\lambda z)\bm k,$$
 即$\lambda\bm a=(\lambda x,\lambda y,\lambda z).$

\end{frame}

\begin{frame}
 (3) 数量积的坐标运算
 \vskip 5pt
 设$\bm a_1=(x_1,y_1,z_1),~\bm a_2=(x_2,y_2,z_2)$,则
 $$\bm a_1\cdot\bm a_2=x_1x_2+y_1y_2+z_1z_2.$$
 \pause 事实上
 \begin{eqnarray*}
 \bm a_1\cdot\bm a_2&=&(x_1\bm i+y_1\bm j+z_1\bm k)\cdot(x_2\bm i+y_2\bm j+z_2\bm k) \\
 &=&x_1x_2\bm i\cdot\bm i+x_1y_2\bm i\cdot\bm j+x_1z_2\bm i\cdot\bm k \\
 && y_1x_2\bm j\cdot\bm i+y_1y_2\bm j\cdot\bm j+y_1z_2\bm j\cdot\bm k \\
 && z_1x_2\bm k\cdot\bm i+z_1y_2\bm k\cdot\bm j+z_1z_2\bm k\cdot\bm k\\
 &=& x_1x_2+y_1y_2+z_1z_2.
 \end{eqnarray*}

\end{frame}


\begin{frame}
 (4) 向量积的坐标运算
 \vskip 5pt
 设$\bm a_1=(x_1,y_1,z_1),~\bm a_2=(x_2,y_2,z_2)$,则
 $$\bm a_1\times\bm a_2=\left(\left|\begin{matrix}y_1&z_1\\y_2&z_2\end{matrix}\right|,~-\left|\begin{matrix}x_1&z_1\\x_2&z_2\end{matrix}\right|,~\left|\begin{matrix}x_1&y_1\\x_2&y_2\end{matrix}\right|\right),$$
 或写成
  $$\bm a_1\times\bm a_2=\left|\begin{matrix}\bm i&\bm j&\bm k\\x_1&y_1&z_1\\x_2&y_2&z_2\end{matrix}\right|.$$
 \pause 事实上
 \begin{eqnarray*}
 \bm a_1\times\bm a_2&=&(x_1\bm i+y_1\bm j+z_1\bm k)\times(x_2\bm i+y_2\bm j+z_2\bm k) \\
 &=&x_1x_2\bm i\times\bm i+x_1y_2\bm i\times\bm j+x_1z_2\bm i\times\bm k \\
 && y_1x_2\bm j\times\bm i+y_1y_2\bm j\times\bm j+y_1z_2\bm j\times\bm k \\
 && z_1x_2\bm k\times\bm i+z_1y_2\bm k\times\bm j+z_1z_2\bm k\times\bm k \\
 &=& \left|\begin{matrix}y_1&z_1\\y_2&z_2\end{matrix}\right|\bm i-\left|\begin{matrix}x_1&z_1\\x_2&z_2\end{matrix}\right|\bm j+\left|\begin{matrix}x_1&y_1\\x_2&y_2\end{matrix}\right|\bm k.
 \end{eqnarray*}
\end{frame}

\begin{frame}
 (5) 混合积的坐标运算
 \vskip 5pt
 设$\bm a_1=(x_1,y_1,z_1),~\bm a_2=(x_2,y_2,z_2),~\bm a_3=(x_3,y_3,z_3)$,则
 $$(\bm a_1,\bm a_2,\bm a_3)=\left|\begin{matrix}x_1&y_1&z_1\\x_2&y_2&z_2\\x_3&y_3&z_3\end{matrix}\right|.$$
  \pause 事实上
 \begin{eqnarray*}
  (\bm a_1,\bm a_2,\bm a_3)&=& (\bm a_1\times\bm a_2)\cdot\bm a_3=\bm a_1\cdot(\bm a_2\times\bm a_3) \\
  &=&(x_1,y_1,z_1)\cdot
  \left(\left|\begin{matrix}y_2&z_2\\y_3&z_3\end{matrix}\right|,~-\left|\begin{matrix}x_2&z_2\\x_3&z_3\end{matrix}\right|,~\left|\begin{matrix}x_2&y_2\\x_3&y_3\end{matrix}\right|\right)\\
  &=&x_1\left|\begin{matrix}y_2&z_2\\y_3&z_3\end{matrix}\right|-y_1\left|\begin{matrix}x_2&z_2\\x_3&z_3\end{matrix}\right|+z_1\left|\begin{matrix}x_2&y_2\\x_3&y_3\end{matrix}\right|\\
  &=&\left|\begin{matrix}x_1&y_1&z_1\\x_2&y_2&z_2\\x_3&y_3&z_3\end{matrix}\right|.
 \end{eqnarray*}
\end{frame}

\begin{frame}
 (6) 向量的长度和两向量夹角的坐标表示
 \vskip 5pt
 设$\bm a=(x,y,z)$, 则
 $$|\bm a|=\pause\sqrt{\bm a\cdot\bm a}=\sqrt{x^2+y^2+z^2}.$$
 \pause 又设$\bm a_1=(x_1,y_1,z_1),~\bm a_2=(x_2,y_2,z_2)$,则由内积定义知
 $$\cos\langle\bm a_1,\bm a_2\rangle=\pause \frac{\bm a_1\cdot\bm a_2}{|\bm a_1||\bm a_2|}=\frac{x_1x_2+y_1y_2+z_1z_2}{\sqrt{x_1^2+y_1^2+z_1^2}\cdot\sqrt{x_2^2+y_2^2+z_2^2}}.$$
 \pause 一些结论:
 \vskip 5pt
 (i) ~~$\bm a_1//\bm a_2$当且仅当$\frac{x_1}{x_2}=\frac{y_1}{y_2}=\frac{z_1}{z_2}$;
 \vskip 5pt
 (ii) ~$\bm a_1\bot\bm a_2$当且仅当$x_1x_2+y_1y_2+z_1z_2=0$;
 \vskip 5pt
 (iii) $\bm a_1,\bm a_2,\bm a_3$共面当且仅当$\left|\begin{matrix}x_1&y_1&z_1\\x_2&y_2&z_2\\x_3&y_3&z_3\end{matrix}\right|=0$.
\end{frame}

\begin{frame}

 \ \ \ \ {\hei 例4} 设向量$\bm a=(x,y,z)$和$x$轴,$y$轴,$z$轴正向的夹角分别是$\alpha$、$\beta$、$\gamma$,则$\alpha$、$\beta$、$\gamma$称为$\bm a$的{\hei 方向角},其余弦值$\cos\alpha,\cos\beta,$ $\cos\gamma$称为$\bm a$的{\hei 方向余弦},试证:
 $$\cos^2\alpha+\cos^2\beta+\cos^2\gamma=1.$$
 \pause
 \ \ \ \ {\hei 证明} 由$\bm a=x\bm i+y\bm j+z\bm k$,得
 $$\bm i\cdot\bm a=x,~~\bm j\cdot\bm a=y,~~\bm k\cdot\bm a=z.$$
 又因为
 $$\bm i\cdot\bm a=|\bm i||\bm a|\cos\langle\bm i,\bm a\rangle=|\bm a|\cos\alpha,$$
 所以
 $$\cos\alpha=\frac{x}{|\bm a|}=\frac{x}{\sqrt{x^2+y^2+z^2}}.$$
 同理可得
 $$\cos\beta=\frac{y}{|\bm a|}=\frac{y}{\sqrt{x^2+y^2+z^2}},~~~~~\cos\gamma=\frac{z}{|\bm a|}=\frac{z}{\sqrt{x^2+y^2+z^2}}.$$

\end{frame}
%-------------------------------------------------frame6---------------------------------------------------
\section[1.6]{1.6 平面方程}
\begin{frame} \frametitle{平面的点法式方程和一般方程}
 \ \ \ \ 在空间直角坐标系$\{O;x,y,z\}$中,设$\pi$为一个平面,我们称与$\pi$垂直的任意非零向量为$\pi$的{\hei 法向量}(或{\hei 法矢})。平面法向量有两种取法,方向相反。
 \pause\vskip 10pt
 \ \ \ \ 设$\pi$的法向量为$\bm n=(A,B,C)$, $P_0(x_0,y_0,z_0)$是$\pi$中的一个固定点,则点$P(x,y,z)$在$\pi$内的充要条件是$\overset{\longrightarrow}{P_0P}\bot\bm n$ (或$\overset{\longrightarrow}{P_0P}\cdot\bm n=0$),\pause 从而得{\hei 平面的点法式方程}
  $$A(x-x_0)+B(y-y_0)+C(z-z_0)=0.$$
 \pause 令$D=-Ax_0-By_0-Cz_0$,得{\hei 平面的一般方程}
 $$Ax+By+Cz+D=0.$$


\end{frame}


%-------------------------------------------------frame7--------------------------------------------------
\begin{frame}
 \ \ \ \ {\hei 定理~1.1} 在空间直角坐标系下,任何平面都可以表示成一个三元一次方程。
 \pause\vskip 10pt
 \ \ \ \ {\hei 定理~1.2} 在空间直角坐标系下,任何一个三元一次方程都表示一个平面,且未知量系数组成的非零向量$(A,B,C)$是该平面的一个法向量。

\end{frame}

\begin{frame}
\ \ \ \ {\hei 定理~1.2} 在空间直角坐标系下,任何一个三元一次方程都表示一个平面,且未知量系数组成的非零向量$(A,B,C)$是该平面的一个法向量。
\pause\vskip 5pt
 \ \ \ \ {\hei 定理~1.2 的证明} 设$P_0(x_0,y_0,z_0)$为满足方程$Ax+By+Cz+D=0$的一个点,
 即
 $$Ax_0+By_0+Cz_0+D=0.$$
 这里$A,B,C$不全为0。设$P(x,y,z)$为满足方程的任意点,即
 $$Ax+By+Cz+D=0$$
 两式相减,得
 $$A(x-x_0)+B(y-y_0)+C(z-z_0)=0$$
 即$\overset{\longrightarrow}{P_0P}\bot\bm n=(A,B,C)$
 \vskip 2pt
 \ \ \ \ 因此所有的点$P(x,y,z)$组成一个平面,且这个平面过点$P_0$,法向量为$\bm n$.
\end{frame}

\begin{frame}{三点式方程}
 \ \ \ \ {\hei 例~4} 设三点$P_1(x_1,y_1,z_1),P_2(x_2,y_2,z_2),P_3(x_3,y_3,z_3)$不共线,证明过这三点的平面方程是
 $$\left|\begin{matrix}x-x_1&y-y_1&z-z_1\\x_2-x_1&y_2-y_1&z_2-z_1\\x_3-x_1&y_3-y_1&z_3-z_1\end{matrix}\right|=0.$$
 \pause
 \ \ \ \ {\hei 证明} 设$P(x,y,z)$是所求平面$\pi$内任意一点,显然$\overset{\longrightarrow}{P_1P},\overset{\longrightarrow}{P_1P_2},$ $\overset{\longrightarrow}{P_1P_3}$共面,即它们的混合积
 $$(\overset{\longrightarrow}{P_1P},\overset{\longrightarrow}{P_1P_2},\overset{\longrightarrow}{P_1P_3})
 =\left|\begin{matrix}x-x_1&y-y_1&z-z_1\\x_2-x_1&y_2-y_1&z_2-z_1\\x_3-x_1&y_3-y_1&z_3-z_1\end{matrix}\right|=0.$$
\end{frame}

\begin{frame}{截距式方程}
\ \ \ \ {\hei 例~5} 求过点$(a,0,0),(0,b,0),(0,0,c)$的平面方程,其中$a,b,c$全不为0.
\pause\vskip 5pt
 \ \ \ \ {\hei 解} 由例4知,所求平面方程为
 $$\left|\begin{matrix}x-a&y&z\\-a&b&0\\-a&0&c\end{matrix}\right|=0,$$
 从而得$bc(x-a)+acy+abz=0$,即
 $$\frac{x}a+\frac{y}b+\frac{z}c=1.$$
 称上式为平面的{\hei 截距式方程},$a,b,c$分别称为平面在$x$轴,$y$轴, $z$轴上的{\hei 截距}。
\end{frame}

\begin{frame}{空间中点到平面的距离公式}\small
 \ \ \ \ {\hei 例~6} 点$P_0(x_0,y_0,z_0)$到平面$\pi:~Ax+By+Cz+D=0$的距离是
 $$d=\frac{|Ax_0+By_0+Cz_0+D|}{\sqrt{A^2+B^2+C^2}}.$$
 \pause
 \ \ \ \ {\hei 证明} 过$P_0$作$\pi$的垂线交$\pi$于点$P_1(x_1,y_1,z_1)$,则$\overset{\longrightarrow}{P_1P_0}//\bm n=(A,B,C)$,于是$\overset{\longrightarrow}{P_1P_0}=\pm d\frac{\bm n}{|\bm n|}$. 从而有
 \begin{eqnarray*}
 x_0-x_1&=&\frac{\pm Ad}{\sqrt{A^2+B^2+C^2}},\\
 y_0-y_1&=&\frac{\pm Bd}{\sqrt{A^2+B^2+C^2}},\\
 z_0-z_1&=&\frac{\pm Cd}{\sqrt{A^2+B^2+C^2}}.
 \end{eqnarray*}
 又因为$Ax_1+By_1+Cz_1+D=0$且$d\geq0$,所以有
 $$d=\frac{|Ax_0+By_0+Cz_0+D|}{\sqrt{A^2+B^2+C^2}}.$$

\end{frame}

\begin{frame}{平面之间的关系}
\ \ \ \ {\hei 定理~1.3} 设$\pi_1,\pi_2$为两个平面,其方程分别为
\begin{eqnarray*}
&&\pi_1:~A_1x+B_1y+C_1z+D_1=0,\\
&&\pi_2:~A_2x+B_2y+C_2z+D_2=0,
\end{eqnarray*}
令$\bm n_1=(A_1,B_1,C_1),~\bm n_2=(A_2,B_2,C_2),$ 则有以下结论:
\pause\vskip 5pt
\ \ \ \ (1) $\pi_1$与$\pi_2$平行$~\Longleftrightarrow~\bm n_1$与$\bm n_2$共线;
\vskip 5pt
\ \ \ \ (2) $\pi_1$与$\pi_2$相交$~\Longleftrightarrow~\bm n_1$与$\bm n_2$不共线;
\vskip 5pt
\ \ \ \ (3) $\pi_1$与$\pi_2$重合$~\Longleftrightarrow~\bm n_1$与$\bm n_2$共线且$\pi_1$与$\pi_2$有公共点。

\end{frame}

\section[1.7]{1.7 直线方程}
\begin{frame}{直线的参数方程和点向式方程}
 \ \ \ \ 在空间直角坐标系$\{O;x,y,z\}$中,与直线$l$平行的任意非零向量称为直线$l$的{\hei 方向向量}。
 \pause\vskip 10pt
 \ \ \ \ 设直线$l$过点$P_0(x_0,y_0,z_0)$,其方向向量为$\bm v=(X,Y,Z)\neq0$,则对直线$l$上任意一点$P(x,y,z)$,都有$\overset{\longrightarrow}{P_0P}//\bm v$,于是存在实数$t$使$\overset{\longrightarrow}{P_0P}=t\bm v$,即$(x-x_0,y-y_0,z-z_0)=(tX,tY,tZ)$,由此可得
 $$\left\{\begin{array}{l}
 x=x_0+tX,\\y=y_0+tY,\\z=z_0+tZ.
 \end{array}\right.$$
 此式称为直线的{\hei 参数方程}。\pause 消去参数$t$,可得
 $$\frac{x-x_0}X=\frac{y-y_0}Y=\frac{z-z_0}Z.$$
 此式称为直线的{\hei 点向式方程} (或{\hei 对称方程})。

\end{frame}

\begin{frame}
 \ \ \ \ {\hei 注} 直线的点向式方程允许分母中某一个或两个为0的情形。例如当$X\neq0,Y=0,Z\neq0$时,即
 $$\frac{x-x_0}X=\frac{y-y_0}0=\frac{z-z_0}Z.$$
 此时理解为
 $$\left\{\begin{array}{l}
 \frac{x-x_0}X=\frac{z-z_0}Z,\\
 y=y_0.
 \end{array}\right.$$
 \pause 当$X\neq0,Y=Z=0$时,即
 $$\frac{x-x_0}X=\frac{y-y_0}0=\frac{z-z_0}0.$$
 此时理解为
 $$\left\{\begin{array}{l}
 y=y_0,\\
 z=z_0.
 \end{array}\right.$$
\end{frame}

\begin{frame}{直线的一般方程}
 \ \ \ \ 设$\pi_1,\pi_2$是两个相交平面,其方程分别为
 \begin{eqnarray*}
&&\pi_1:~A_1x+B_1y+C_1z+D_1=0,\\
&&\pi_2:~A_2x+B_2y+C_2z+D_2=0,
\end{eqnarray*}
则它们的交线可表示为
$$\left\{\begin{array}{l}
 A_1x+B_1y+C_1z+D_1=0,\\
 A_2x+B_2y+C_2z+D_2=0,
 \end{array}\right.$$
此式称为直线的{\hei 一般方程}。\pause 此直线的方向向量可以取为
$$\bm v=\bm n_1\times\bm n_2=\left(\left|\begin{matrix}B_1&C_1\\B_2&C_2\end{matrix}\right|,\left|\begin{matrix}C_1&A_1\\C_2&A_2\end{matrix}\right|,
\left|\begin{matrix}A_1&B_1\\A_2&B_2\end{matrix}\right|\right).$$
这里$\bm n_1=(A_1,B_1,C_1),~\bm n_2=(A_2,B_2,C_2).$
\end{frame}

\begin{frame}{一般方程与点向式方程之间的相互转化}
 \ \ \ \ 利用直线的点向式方程很容易得到直线的一般方程。反之,在两平面交线$l$上任取一点$P_0(x_0,y_0,z_0)$,可得到直线$l$的点向式方程
 $$\frac{x-x_0}{\left|\begin{matrix}B_1&C_1\\B_2&C_2\end{matrix}\right|}=\frac{y-y_0}{\left|\begin{matrix}C_1&A_1\\C_2&A_2\end{matrix}\right|}
 =\frac{z-z_0}{\left|\begin{matrix}A_1&B_1\\A_2&B_2\end{matrix}\right|}.$$
\end{frame}

\begin{frame}{直线的两点式方程}
\ \ \ \ {\hei 例 3} 设直线$l$过两个不同点$P_1(x_1,y_1,z_1)$和$P_2(x_2,y_2,z_2)$,求$l$的方程。
 \pause\vskip 5pt
 \ \ \ \ {\hei 解} 显然直线的方向向量可取为
 $$\overset{\longrightarrow}{P_1P_2}=(x_2-x_1,y_2-y_1,z_2-z_1).$$
 又因为$l$过点$P_1$,所以直线$l$的方程为
 $$\frac{x-x_1}{x_2-x_1}=\frac{y-y_1}{y_2-y_1}=\frac{z-z_1}{z_2-z_1}.$$
 此式称为直线的{\hei 两点式方程}。
\end{frame}

\begin{frame}{直线之间的关系}
 \ \ \ \ {\hei 定理~1.4} 设$l_1,l_2$为分别过$P_1(x_1,y_1,z_1),P_2(x_2,y_2,z_2)$的两条直线,其方程分别为
 \begin{eqnarray*}
 &&l_1:~\frac{x-x_1}{X_1}=\frac{y-y_1}{Y_1}=\frac{z-z_1}{Z_1},\\
 &&l_2:~\frac{x-x_2}{X_2}=\frac{y-y_2}{Y_2}=\frac{z-z_2}{Z_2},
 \end{eqnarray*}
 令$\bm v_1=(X_1,Y_1,Z_1),~\bm v_2=(X_2,Y_2,Z_2)$,则有以下结论:
 \pause\vskip 5pt
 \ \ \ \ (1) $l_1$与$l_2$平行~$\Longleftrightarrow$~$\bm v_1$与$\bm v_2$共线;
  \vskip 5pt
 \ \ \ \ (2) $l_1$与$l_2$重合~$\Longleftrightarrow$~$\bm v_1,\bm v_2,\overset{\longrightarrow}{P_1P_2}$三个向量共线;
  \vskip 5pt
 \ \ \ \ (3) $l_1$与$l_2$相交~$\Longleftrightarrow$~$\bm v_1,\bm v_2$不共线且混合积$(\overset{\longrightarrow}{P_1P_2},\bm v_1,\bm v_2)=0$;
 \vskip 5pt
 \ \ \ \ (4) $l_1$与$l_2$为异面直线~$\Longleftrightarrow$~$(\overset{\longrightarrow}{P_1P_2},\bm v_1,\bm v_2)\neq0$.
\end{frame}


\begin{frame}{直线与平面的关系}
 \ \ \ \ {\hei 定理~1.5} 设直线$l$的方程为$\frac{x-x_0}X=\frac{y-y_0}Y=\frac{z-z_0}Z$,方向向量为$\bm v=(X,Y,Z)\neq0$. 又设平面$\pi$的方程为$Ax+By+Cz+D=0$,法向量为$\bm n=(A,B,C)\neq0$. 则有以下结论:
 \pause\vskip 5pt
 \ \ \ \ (1) $l$与$\pi$平行~$\Longleftrightarrow$~$AX+BY+CZ=0$;
 \vskip 5pt
 \ \ \ \ (2) $l$在$\pi$内~$\Longleftrightarrow$~$AX+BY+CZ=0$且$Ax_0+By_0+Cz_0+D=0$;
 \vskip 5pt
 \ \ \ \ (3) $l$与$\pi$相交~$\Longleftrightarrow$~$AX+BY+CZ\neq0$;
 \vskip 5pt
 \ \ \ \ (4) $l$与$\pi$垂直相交~$\Longleftrightarrow$~$\bm v$与$\bm n$共线。
\end{frame}

\begin{frame}{点到直线的距离}
 \ \ \ \ {\hei 定理~1.6} 设直线$l:\frac{x-x_0}X=\frac{y-y_0}Y=\frac{z-z_0}Z$过点$P_0(x_0,y_0,z_0)$,方向向量为$\bm v=(X,Y,Z)$. 则点$P_1(x_1,y_1,z_1)$到直线$l$的距离为$d=\frac{|\bm v\times\overset{\longrightarrow}{P_0P_1}|}{|\bm v|}$.
 \pause\vskip 5pt
 \ \ \ \ {\hei 证明} 设$\bm v=\overset{\longrightarrow}{P_0P_2}$是$l$上的一个向量,如图所示。
 \setlength{\unitlength}{2cm}\begin{center}
 \begin{picture}(1,1)
 \put(0,0){\line(1,0){2}}
 \put(0,0){\vector(1,0){1}}
 \put(0,0){\line(1,2){0.5}}
 \put(1,0){\line(1,2){0.5}}\put(0.5,1){\line(1,0){1}}\put(0.5,1){\line(0,-1){1}}

 \scriptsize{
 \put(-0.1,-0.14){$P_0$}\put(0.9,-0.14){$P_2$}\put(2.1,-0.05){$l$}\put(0.4,1.05){$P_1$}\put(1.4,1.05){$P_3$}\put(0.4,.4){$d$}
 }
 \end{picture}\end{center}

 \pause 平行四边形$P_0P_1P_3P_2$的面积为$S=|\bm v\times\overset{\longrightarrow}{P_0P_1}|$. 已知$S=|\bm v|d$,所以有
 $$d=\frac{|\bm v\times\overset{\longrightarrow}{P_0P_1}|}{|\bm v|}.$$

\end{frame}

\begin{frame}{异面直线之间的距离}
 \ \ \ \ {\hei 定理~1.7} 两条异面直线$l_i:~\frac{x-x_i}{X_i}=\frac{y-y_i}{Y_i}=\frac{z-z_i}{Z_i},~i=1,2$ 之间的距离为
 $$d=\frac{|(\overset{\longrightarrow}{P_1P_2},\bm v_1,\bm v_2)|}{|\bm v_1\times\bm v_2|},$$
 其中$P_i(x_i,y_i,z_i),~\bm v_i=(X_i,Y_i,Z_i)$.
 \pause\vskip 5pt
 \ \ \ \ {\hei 证明} 设平面$\pi$过直线$l_1$,法向量取为$\bm n=\bm v_1\times\bm v_2$,则$d$就是$\overset{\longrightarrow}{P_1P_2}$在法向量$\bm n$上的投影长度。因此结论成立。

\end{frame}


%=====================================================frame8=========================================


\begin{frame}
\begin{center}
{\textcolor[rgb]{0.50,0.00,1.00}{\textbf{\xiaoerhao{Thanks for your attention!}}}}\bigskip
\end{center}
\end{frame}
\end{CJK}
\end{document}


