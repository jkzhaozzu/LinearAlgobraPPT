\documentclass[compress,mathserif,cjk]{beamer}
\usepackage{mathrsfs}
\usepackage{color}
\usepackage{CJK}
\usepackage{amssymb}
\usepackage{amsmath}
\usepackage{extarrows}
\usepackage{ulem}
\usepackage{latexsym}
\usepackage{pmat}

% Copyright 2003 by Till Tantau <tantau@cs.tu-berlin.de>.
%
% This program can be redistributed and/or modified under the terms
% of the LaTeX Project Public License Distributed from CTAN
% archives in directory macros/latex/base/lppl.txt.

%
% The purpose of this example is to show how \part can be used to
% organize a lecture.
%
\usetheme{Warsaw}
  % 可供选择的主题参见 beameruserguide.pdf, 第 134 页起
  % 无导航条的主题: Bergen, Boadilla, Madrid, Pittsburgh, Rochester;
  % 有树形导航条的主题: Antibes, JuanLesPins, Montpellier;
  % 有目录竖条的主题: Berkeley, PaloAlto, Goettingen, Marburg, Hannover;
  % 有圆点导航条的主题: Berlin, Dresden, Darmstadt, Frankfurt, Singapore, Szeged;
  % 有节与小节导航条的主题: Copenhagen, Luebeck, Malmos, Warsaw

%  \setbeamercovered{transparent}
% 如果取消上一行的注解 %, 就会使得被覆盖部分变得透明(依稀可见)

\usepackage[english]{babel}
\usepackage[latin1]{inputenc}
\usepackage{CJK}

%\usepackage{beamerthemesplit}
%\usepackage{beamerthemeshadow,color}
\usepackage{pgf,pgfarrows,pgfnodes,pgfautomata,pgfheaps,pgfshade}
\usepackage{amsmath,amssymb}
\usepackage{bm}
\usepackage{colortbl}

\graphicspath{{images/}}         %% 图片路径. 本文的图片都放在这个文件夹里了.
\DeclareGraphicsRule{*}{mps}{*}{} %% 使 pdflatex 可以纳入 metapost 做的图片.
\renewcommand{\div}{\operatorname{div}}
\renewcommand{\raggedright}{\leftskip=0pt \rightskip=0pt plus 0cm}
\raggedright %% 中文对齐

%==============自定义: 逐个 item 高亮(\hilite), 或"高黑"(\hidark)==================%
\def\hilite<#1>{%
\temporal<#1>{\color{blue!35}}{\color{magenta}}%
{\color{blue!75}}}
\def\hidark<#1>{%
\temporal<#1>{\color{black!35}}{\color{magenta}}%
{\color{black}}}

\newcolumntype{H}{>{\columncolor{blue!20}}c!{\vrule}}
\newcolumntype{H}{>{\columncolor{blue!20}}c}  %% 表格设置
%==================================参考文献==============================================================
\newcommand{\upcite}[1]{\textsuperscript{\cite{#1}}}  %自定义命令\upcite, 使参考文献引用以上标出现
\bibliographystyle{plain}
%=========================================================================================

\def\colorb{\textcolor[rgb]{0.00,0.00,1.00}}
\def\colorg{\textcolor[rgb]{0.00,1.00,0.00}}
\def\colorr{\textcolor[rgb]{1.00,0.00,0.00}}
\newcommand\hdim{\dim_{\mathrm H}}
\newtheorem{lem}{Lemma}[section]
\newtheorem{nt}[lem]{Notation}
\newtheorem{dfn}[lem]{Definition}
\newtheorem{pro}[lem]{Proposition}
\newtheorem{thm}[lem]{Theorem}
\newtheorem{exa}[lem]{Example}
\newtheorem{cor}[lem]{Corollary}
\theoremstyle{remark}
\newtheorem*{rem}{Remark}
\numberwithin{equation}{section}
\def\N{\mathbb N}
\def\Q{\mathbb Q}
\def\R{\mathbb R}
\def\Z{\mathbb Z}
\def\vep{\varepsilon}
%=========================================================================================
%\setbeamercovered{transparent}
%
% The following info should normally be given in you main file:
%

%%%%%%%%%%%%%%%%%%%%%%%%%%%%%%%%%%%%%%%%%%%%%%%%%%%%%%%%%%%%%%%%%%%%%%%%%%%%%%%%%%%%%%%%%%%%%%%%%%%%%%%%%
%                                           定制幻灯片---重定义字体、字号命令                           %
%%%%%%%%%%%%%%%%%%%%%%%%%%%%%%%%%%%%%%%%%%%%%%%%%%%%%%%%%%%%%%%%%%%%%%%%%%%%%%%%%%%%%%%%%%%%%%%%%%%%%%%%%
\newcommand{\song}{\CJKfamily{song}}    % 宋体   (Windows自带simsun.ttf)
\newcommand{\fs}{\CJKfamily{fs}}        % 仿宋体 (Windows自带simfs.ttf)
\newcommand{\kai}{\CJKfamily{kai}}      % 楷体   (Windows自带simkai.ttf)
\newcommand{\hei}{\bf}      % 黑体   (Windows自带simhei.ttf)
\newcommand{\li}{\CJKfamily{li}}        % 隶书   (Windows自带simli.ttf)
\newcommand{\you}{\CJKfamily{you}}      % 幼圆   (Windows自带simyou.ttf)
\newcommand{\chuhao}{\fontsize{42pt}{\baselineskip}\selectfont}     % 字号设置
\newcommand{\xiaochuhao}{\fontsize{36pt}{\baselineskip}\selectfont} % 字号设置
\newcommand{\yichu}{\fontsize{32pt}{\baselineskip}\selectfont}      % 字号设置
\newcommand{\yihao}{\fontsize{28pt}{\baselineskip}\selectfont}      % 字号设置
\newcommand{\erhao}{\fontsize{21pt}{\baselineskip}\selectfont}      % 字号设置
\newcommand{\xiaoerhao}{\fontsize{18pt}{\baselineskip}\selectfont}  % 字号设置
\newcommand{\sanhao}{\fontsize{15.75pt}{\baselineskip}\selectfont}  % 字号设置
\newcommand{\sihao}{\fontsize{14pt}{\baselineskip}\selectfont}      % 字号设置
\newcommand{\xiaosihao}{\fontsize{12pt}{\baselineskip}\selectfont}  % 字号设置
\newcommand{\wuhao}{\fontsize{10.5pt}{\baselineskip}\selectfont}    % 字号设置
\newcommand{\xiaowuhao}{\fontsize{9pt}{\baselineskip}\selectfont}   % 字号设置
\newcommand{\liuhao}{\fontsize{7.875pt}{\baselineskip}\selectfont}  % 字号设置
\newcommand{\qihao}{\fontsize{5.25pt}{\baselineskip}\selectfont}    % 字号设置

%======================= 标题名称中文化 ============================%
\newtheorem{dingyi}{\hei 定义~}[section]
\newtheorem{dingli}{\hei 定理~}[section]
\newtheorem{yinli}[dingli]{\hei 引理~}
\newtheorem{tuilun}[dingli]{\hei 推论~}
\newtheorem{mingti}[dingli]{\hei 命题~}
%%%%%%%%%%%%%%%%%%%%%%%%%%%%%%%%%%%%%%%%%%%%%%%%%%%%%%%%%%%%%%%%%%%%%%%%%%%%%%%%%%%%%%%

% \usepackage{beamerthemesplit} // Activate for custom appearance


\title{\textsc{第4章\ \ \ 线性方程组}}
\author{郑州大学数学与统计学院 线性代数教研室}
\date{}

\begin{document}
\begin{CJK}{UTF8}{gbsn}
\frame{\titlepage}




\section[4.2]{4.2 $n$维向量空间与欧氏空间}




\section[4.3]{4.3 $P^n$中向量的线性相关性}








\section[4.4]{4.4 向量组的秩和矩阵的秩}




\begin{frame}
 \ \ \ \ 利用秩的概念,可以更精确的描述可逆矩阵和退化矩阵
 \pause\vskip 5pt
 \ \ \ \ {\hei 定理~4.12} 方阵$\bm A=(a_{ij})_{nn}$可逆的充要条件是$\bm A$的行向量组(或列向量组)是线性无关的(即$\bm A$的秩$r(\bm A)=n$).
 \pause\vskip 10pt
 \ \ \ \ {\hei 定理~4.12'} 方阵$\bm A=(a_{ij})_{nn}$退化的充要条件是$\bm A$的行向量组(或列向量组)是线性相关的(即$\bm A$的秩$r(\bm A)<n$).
  \pause\vskip 10pt
 \ \ \ \ {\hei 定义~4.16} 若$n$阶矩阵$\bm A$的秩为$n$,则称$\bm A$为{\hei 满秩矩阵}。
\end{frame}

\begin{frame}{$k$阶子式}
 \ \ \ \ {\hei 定义~4.17} 在矩阵$\bm A=(a_{ij})_{mn}$中任取$k$行$i_1,i_2,\cdots,i_k$及$k$列$j_1,j_2,$ $\cdots,j_k$,位于这$k$行$k$列交叉点上的$k^2$个元素按原来顺序排列成的一个$k$行列式,称为$\bm A$的一个{\hei $k$阶子式},记为$\bm A\left[\begin{matrix}i_1&i_2&\cdots&i_k\\j_1&j_2&\cdots&j_k\end{matrix}\right]$.
 \pause\vskip 10pt
 \ \ \ \ {\hei 例~4} 设
 $$\bm A=\left[\begin{matrix}1&3&0&1\\2&0&2&1\\3&0&0&-2\end{matrix}\right],$$
 求$\bm A\left[\begin{matrix}2&3\\1&2\end{matrix}\right]$和$\bm A\left[\begin{matrix}1&2&3\\2&3&4\end{matrix}\right]$.
\end{frame}

\begin{frame}
 \ \ \ \ 可以利用$k$阶子式来描述矩阵的秩
 \vskip 5pt
 \ \ \ \ {\hei 定理~4.13} 矩阵$\bm A$的秩$r(\bm A)=r$的充要条件是$\bm A$有非零的$r$阶子式,且$\bm A$的任意$r+1$阶子式都是零。
 \pause\vskip 10pt
 \ \ \ \ {\hei 证明} 必要性. 由$r(\bm A)=r$知,$\bm A$中有$r$个行向量线性无关,取出来作成一个$r\times n$矩阵$\bm A_1$,则$r(\bm A_1)=r$. 于是$\bm A_1$有$r$个列向量线性无关,再取出来作成一个$r\times r$ 矩阵$\bm A_2$,则$r(\bm A_2)=r$. 由定理4.12知$|\bm A_2|\neq0$,因此$|\bm A_2|$就是$\bm A$的一个非零的$r$阶子式。
 \pause\vskip 5pt
 \ \ \ \ 如果$\bm A$有一个非零的$r+1$阶子式$|\bm A_3|$,则$\bm A_3$的行向量组线性无关,由4.3节例10知,添加分量得到$\bm A$的$r+1$个$n$维行向量仍然线性无关。因此$\bm A$中有$r+1$个线性无关的行向量,这是与$r(\bm A)=r$矛盾。因此$\bm A$中没有非零的$r+1$阶子式。

\end{frame}
\begin{frame}\small
 \ \ \ \ 充分性. 设$\bm A$有非零的$r$阶子式,而所有$r+1$阶子式都是零。由行列式展开定理知,$\bm A$的所有大于$r$阶的子式都是0. 不失一般性,设
 $$|\bm B_1|=\left|\begin{matrix}a_{11}&a_{12}&\cdots&a_{1r}\\a_{21}&a_{22}&\cdots&a_{2r}\\ \vdots&\vdots&&\vdots\\a_{r1}&a_{r2}&\cdots&a_{rr}\end{matrix}\right|\neq0.$$
 由定理4.12知$r(\bm B_1)=r$. \pause 令
 $$\bm B_2=\left[\begin{matrix}a_{11}&a_{12}&\cdots&a_{1r}&\cdots&a_{1n}\\a_{21}&a_{22}&\cdots&a_{2r}&\cdots&a_{2n}\\ \vdots&\vdots&&\vdots&&\vdots\\a_{r1}&a_{r2}&\cdots&a_{rr}&\cdots&a_{rn}\end{matrix}\right|.$$
 由4.3节例10知,$\bm B_2$的$r$个行向量也线性无关,故$r(\bm B_2)=r$.
 \pause\vskip 5pt
 \ \ \ \ 显然$r(\bm A)\geq r(\bm B_2)$,故$r(\bm A)\geq r$. 如果$r(\bm A)>r$,则由前面已证明的结论知,$\bm A$有大于$r$阶的非零子式,矛盾。故$r(\bm A)=r.\hfill\Box$
\end{frame}

\begin{frame}
 \ \ \ \ {\hei 例~5} 求矩阵$\bm A$的秩:
 $$\bm A=\left[\begin{matrix}1&3&0&1\\2&0&2&1\\3&0&0&-2\end{matrix}\right].$$
 \pause\vskip 10pt
 \ \ \ \ {\hei 例~6} 求向量组$\bm\alpha_1=(1,0,-1),~\bm\alpha_2=(0,2,1),~\bm\alpha_3=(4,4,-2),$ $\bm\alpha_4=(2,6,1)$的秩。
\end{frame}
\begin{frame}
 \ \ \ \ {\hei 定理~4.14} 齐次线性方程组
 \begin{equation*}
 \left\{\begin{array}{rcl}
 a_{11}x_1+a_{12}x_2+\cdots+a_{1n}x_n&=&0,\\
 a_{21}x_1+a_{22}x_2+\cdots+a_{2n}x_n&=&0,\\
 &\vdots&\\
 a_{n1}x_1+a_{n2}x_2+\cdots+a_{nn}x_n&=&0,
 \end{array}\right.
 \end{equation*}
 有非零解(只有零解)的充要条件是系数矩阵$\bm A$是退化的(可逆的)。
 \end{frame}
\begin{frame}
 \ \ \ \ 至此,$n$阶矩阵$\bm A$可逆的等价说法有
 \vskip 5pt
 \ \ \ \ (1) 存在$n$阶矩阵$\bm B$,使$\bm{AB}=\bm{BA}=\bm E$;
 \vskip 5pt
 \ \ \ \ (2) $|\bm A|\neq0$;
 \vskip 5pt
 \ \ \ \ (3) $\bm A$是非退化矩阵;
 \vskip 5pt
 \ \ \ \ (4) $\bm A$的行(列)向量组线性无关;
 \vskip 5pt
 \ \ \ \ (5) $\bm A$是满秩矩阵;
 \vskip 5pt
 \ \ \ \ (6) 齐次线性方程组$\bm{AX}=\bm 0$只有零解;
 \vskip 5pt
 \ \ \ \ (7) $\bm A$可以写成一系列初等矩阵的乘积;
  \vskip 5pt
 \ \ \ \ (8) $\bm A$的标准形是单位矩阵$\bm E$.
\end{frame}
\setcounter{section}{7}
\section[4.5]{4.5 线性方程组的有解判定定理}

\begin{frame}
 \ \ \ \ 本节我们利用秩来讨论一般线性方程组
 \begin{equation*}
 \left\{\begin{array}{rcl}
 a_{11}x_1+a_{12}x_2+\cdots+a_{1n}x_n&=&b_1,\\
 a_{21}x_1+a_{22}x_2+\cdots+a_{2n}x_n&=&b_2,\\
 &\vdots&\\
 a_{m1}x_1+a_{m2}x_2+\cdots+a_{mn}x_n&=&b_m,\\
 \end{array}\right.
 \end{equation*}
 有解的判定定理及解的个数问题。上面方程组的向量形式为
 $$x_1\bm\alpha_1+x_2\bm\alpha_2+\cdots +x_n\bm\alpha_n=\bm\beta,$$
 其中$\bm\alpha_1,\bm\alpha_2,\cdots,\bm\alpha_n$是系数矩阵的列向量,$\bm\beta$是常数项列向量。
\end{frame}

\begin{frame}
 \ \ \ \ {\hei 引理} 设向量组(I)与向量组(II)的秩相等,且(I)可以由(II)线性表出,则(I)与(II)等价。
 \pause\vskip 10pt
 \ \ \ \ {\hei 定理~4.15} 线性方程组$\bm{AX}=\bm B$有解的充要条件是系数矩阵$\bm A$的秩等于增广矩阵$(\bm A,\bm B)$的秩。
 \pause\vskip 10pt
 \ \ \ \ {\hei 定理~4.16} 设线性方程组$\bm{AX}=\bm B$的系数矩阵$\bm A$为$m\times n$矩阵,增广矩阵为$(\bm A,\bm B)$,则
 \vskip 5pt
 \ \ \ \ (1) 当$r(\bm A)=r(\bm A,\bm B)=n$时,方程组有唯一解;
 \vskip 5pt
 \ \ \ \ (2)  当$r(\bm A)=r(\bm A,\bm B)<n$时,方程组有无穷多个解;
 \vskip 5pt
 \ \ \ \ (3)  当$r(\bm A)\neq r(\bm A,\bm B)$~(这时必有$r(\bm A,\bm B)=r(\bm A)+1$)时,方程组无解。
\end{frame}

\begin{frame}
 \ \ \ \ {\hei 例~1} 设$\bm A$为$m\times n$矩阵,$\bm B$为$m$维列向量。如果$\bm A$的秩$r(\bm A)=m$,则线性方程组$\bm{AX}=\bm B$必有解。
 \pause\vskip 10pt
 \ \ \ \ {\hei 例~2} 解方程组
 $$\left\{\begin{array}{l}
 \lambda x_1+x_2+x_3=1,\\x_1+\lambda x_2+x_3=\lambda,\\x_1+x_2+\lambda x_3=\lambda^2.
 \end{array}\right.$$


\end{frame}


\section[4.6]{4.6 线性方程组解的结构}

\setcounter{section}{4}
\setcounter{equation}{17}
\begin{frame}{齐次线性方程组解的结构}
\ \ \ \ 首先讨论齐次线性方程组
 \begin{equation}\label{e418}
 \left\{\begin{array}{rcl}
 a_{11}x_1+a_{12}x_2+\cdots+a_{1n}x_n&=&0,\\
 a_{21}x_1+a_{22}x_2+\cdots+a_{2n}x_n&=&0,\\
 &\vdots&\\
 a_{m1}x_1+a_{m2}x_2+\cdots+a_{mn}x_n&=&0,\\
 \end{array}\right.
 \end{equation}
 \pause
 \ \ \ \ {\hei 定义~4.18} 齐次线性方程组解空间的一个基(即一个极大线性无关组)称为它的一个{\hei 基础解系}。如果齐次线性方程组只有零解,则称它没有基础解系。
 \pause\vskip 10pt
 \ \ \ \ 易见,一个齐次线性方程组任意两个基础解系都是等价的。
\end{frame}

\begin{frame}
\ \ \ \ {\hei 定理~4.17} 设$\bm A$是$m\times n$矩阵,则齐次线性方程组$\bm{AX}=\bm0$有基础解系的充要条件是其系数矩阵$\bm A$的秩$r(\bm A)=r<n$. 这时它的任意一个基础解系有$n-r$个解向量。
\pause\vskip 10pt
 \ \ \ \ {\hei 证明} 必要性. 当$r(\bm A)=n$时,由定理4.16知$\bm{AX}=\bm0$有唯一解,即只有零解。这时它没有基础解系。所以,$\bm{AX}=\bm0$有基础解系的话,必有$r<n$.
 \pause\vskip 5pt
 \ \ \ \ 充分性. 当$r<n$时由定理4.16知$\bm{AX}=\bm0$有无穷多解,因而有基础解系。
 \end{frame}

\begin{frame}
 \ \ \ \ 下设$r<n$. 适当调整未知量次序,$\bm{AX}=\bm0$可用初等变换化成阶梯形方程组(去掉$0=0$):
 \begin{equation}\label{e419}
 \left\{\begin{array}{rcl}
 c_{11}x_1+c_{12}x_2+\cdots+c_{1r}x_r+\cdots+c_{1n}x_n&=&0,\\
 c_{22}x_2+\cdots+c_{2r}x_r+\cdots+c_{2n}x_n&=&0,\\
 &\vdots&\\
 c_{rr}x_r+\cdots+c_{rn}x_n&=&0,
 \end{array}\right.
 \end{equation}
 其中$c_{ii}\neq0,~i=1,2,\cdots,r,~r<n$.
 \vskip 5pt
 \ \ \ \  进一步作初等变换,并取$x_{r+1},\cdots,x_n$为自由未知量,令$x_{r+1}=c_1,\cdots,x_n=c_{n-r}$,其中$c_1,\cdots,c_{n-r}$为任意常数,得
\end{frame}

\begin{frame}\small
 \begin{equation}\label{e420}
 \left\{\begin{array}{rcl}
 x_1&=&d_{1,r+1}c_1+d_{1,r+2}c_2+\cdots+d_{1n}c_{n-r},\\
 x_2&=&d_{2,r+1}c_1+d_{2,r+2}c_2+\cdots+d_{2n}c_{n-r},\\
 &\vdots& \\
 x_r&=&d_{r,r+1}c_1+d_{r,r+2}c_2+\cdots+d_{rn}c_{n-r},\\
 x_{r+1}&=&c_1, \\
 &\vdots& \\
 x_n&=&c_{n-r},
 \end{array}\right.
 \end{equation}
 把(\ref{e420})写成向量形式,得
 \begin{equation}\label{e421}
 \left[\begin{matrix}x_1\\\vdots\\x_r\\x_{r+1}\\x_{r+2}\\\vdots\\x_n\end{matrix}\right]=c_1\left[\begin{matrix}d_{1,r+1}\\\vdots\\d_{r,r+1}\\1\\0\\\vdots\\0\end{matrix}\right]
 +c_2\left[\begin{matrix}d_{1,r+2}\\\vdots\\d_{r,r+2}\\0\\1\\\vdots\\0\end{matrix}\right]+\cdots+c_{n-r}\left[\begin{matrix}d_{1n}\\\vdots\\d_{rn}\\0\\0\\\vdots\\1\end{matrix}\right]
 \end{equation}
\end{frame}

\begin{frame}
 由此看出,$\bm{AX}=\bm0$的任意一个解都是向量组
 $$\bm\eta_1=\left[\begin{matrix}d_{1,r+1}\\\vdots\\d_{r,r+1}\\1\\0\\\vdots\\0\end{matrix}\right],
 \bm\eta_2=\left[\begin{matrix}d_{1,r+2}\\\vdots\\d_{r,r+2}\\0\\1\\\vdots\\0\end{matrix}\right],\cdots,
 \bm\eta_{n-r}=\left[\begin{matrix}d_{1n}\\\vdots\\d_{rn}\\0\\0\\\vdots\\1\end{matrix}\right]$$
 的线性组合,并且显然它们是线性无关的,因此是解空间的一个基础解系,且有$n-r$个向量。
\end{frame}

\begin{frame}
\ \ \ \ {\hei 推论~1} 设$\bm X_1,\bm X_2,\cdots,\bm X_{n-r}$是方程组$\bm{AX}=\bm0$的任意一个基础解系,则$\bm{AX}=\bm0$的全部解可以写成$c_1\bm X_1+c_2\bm X_2+\cdots+c_{n-r}\bm X_{n-r}$,其中$c_1,c_2,\cdots,c_{n-r}$为任意常数。
\pause\vskip 10pt
\ \ \ \ {\hei 推论~2} 设$\bm A$是$m\times n$矩阵,齐次线性方程组$\bm{AX}=\bm0$的任意$n-r(\bm A)$个线性无关的解向量都构成一个基础解系。
\pause\vskip 10pt
\ \ \ \ 设$V$为$P^n$的子空间,称$V$的极大线性无关组$\bm\alpha_1,\bm\alpha_2,\cdots,\bm\alpha_s$为$V$ 的一个基,$s$称为$V$的维数,记dim$V=s$. 因此,齐次线性方程组$\bm{AX}=\bm0$的解空间的维数为$n-r$,其中$r=r(\bm A)$.
\pause\vskip 10pt
\ \ \ \ {\hei 例~1} 求
$$\left\{\begin{array}{r}
4x_1+2x_2+x_3+2x_4+3x_5=0,\\4x_1+2x_2+x_3+2x_4+x_5=0,\\2x_1+x_2-2x_3+3x_4+2x_5=0,
\end{array}\right.$$
的基础解系。
\end{frame}

\begin{frame}{非齐次线性方程组解的结构}
\ \ \ \ $\bm{AX}=\bm B$的解与$\bm{AX}=0$的解有密切联系。称齐次线性方程组$\bm{AX}=0$为非齐次线性方程组$\bm{AX}=\bm B$的{\hei 导出线性方程组},简称{\hei 导出组}。

\pause\vskip 5pt
\ \ \ \ 不难看出,方程组$\bm{AX}=\bm B$与导出组$\bm{AX}=0$的解有以下关系:
\vskip 5pt
\ \ \ \ (1) $\bm{AX}=\bm B$的任意两个解之差是导出组$\bm{AX}=0$的解;
\vskip 5pt
\ \ \ \ (2) $\bm{AX}=\bm B$的任意一个解与导出组$\bm{AX}=0$的任意一个解之和仍是$\bm{AX}=\bm B$的解。
\pause\vskip 5pt
因此有
\vskip 1pt
\ \ \ \ {\hei 定理~4.18} 设$\bm X_0$为线性方程组$\bm{AX}=\bm B$的一个特解,$\bm X_1,\bm X_2,$ $\cdots,\bm X_{n-r}$ 是导出组$\bm{AX}=0$的任意一个基础解系,则$\bm{AX}=\bm B$ 的解集合为
$$\big\{\bm X_0+c_1\bm X_1+c_2\bm X_2+\cdots+c_{n-r}\bm X_{n-r}|c_1,c_2,\cdots,c_{n-r}\mbox{为任意常数}\big\}.$$

\end{frame}
\begin{frame}
\ \ \ \ {\hei 例~2} 解方程组
$$\left\{\begin{array}{rcl}
x_1+2x_2-x_3+2x_4&=&1,\\2x_1+4x_2+x_3+x_4&=&5,\\-x_1-2x_2-2x_3+x_4&=&-4.
\end{array}\right.$$
\pause\vskip 10pt
\ \ \ \ {\hei 例~3} 解方程组
$$\left\{\begin{array}{rcl}
x_1-2x_2+2x_3+3x_4&=&-5,\\2x_1-x_2+5x_3+6x_4&=&-5,\\2x_1-6x_2+4x_3+7x_4&=&-14,\\4x_1-5x_2+9x_3+12x_4&=&-15.
\end{array}\right.$$
\end{frame}
\begin{frame}
\ \ \ \ {\hei 例~4} 设$\bm{AX}=\bm B$为非齐次线性方程组,$r(\bm A)=2$, $\bm X$为4维列向量。已知$\bm\alpha_1=(1,0,0,0)',~\bm\alpha_2=(1,1,1,1)',~\bm\alpha_3=(1,2,3,4)'$都是$\bm{AX}=\bm B$的解,求该线性方程组的通解。
\end{frame}


%=====================================================frame8=========================================


\begin{frame}
\begin{center}
{\textcolor[rgb]{0.50,0.00,1.00}{\textbf{\xiaoerhao{Thanks for your attention!}}}}\bigskip
\end{center}
\end{frame}
\end{CJK}
\end{document}


